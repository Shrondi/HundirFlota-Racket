\section{Descripción modular del código}

\subsection{Estructuras de datos}

\subsubsection{Barcos}

\begin{frame}
    \frametitle{\insertsection}
    \framesubtitle{\hskip30pt \insertsubsection}

    \begin{block}{Barcos}
        \begin{itemize}
            \item Vectores de vectores (matrices)
            \item Cada fila representa a un \textcolor{blue}{tipo de barco} (portaaviones, acorazado, crucero, destructor o submarino).
            \item Las columnas son los siguientes campos por cada tipo de barco:
            \begin{itemize}
                \item ID: número para identificar el tipo de barco
                \item Nombre: cadena con el nombre del tipo
                \item Número de barcos (sin colocar): número de barcos en total
                \item Tamaño del barco: número de casillas que ocupan
                \item Color: cadena con el color que son representados
                \item Barcos restantes (sin hundir): número de barcos a flote
                \item Lista con el tamaño de cada barco: controla el tamaño de cada barco
            \end{itemize}
        \end{itemize}
    \end{block}
\end{frame}


\subsubsection{Tableros/flota}

\begin{frame}
    \frametitle{\insertsection}
    \framesubtitle{\hskip30pt \insertsubsection}

    \begin{block}{Tableros/flota}
        \begin{itemize}
            \item Vectores de vectores (matrices)
            \item Representan \textcolor{blue}{el tablero de cada jugador.}
            \item Guardan la información de la colocación de los barcos. 
            \item El agua es representada con un 0 mientras que cada barco es una lista:
            \begin{itemize}
                \item ID: número del tipo de barco.
                \item SubID: número que identifica al barco dentro de su mismo tipo.
                \item Posición proa: sublista con las coordenadas de la casilla dónde se sitúa la proa del barco.
                \item Tamaño del barco: número que indica el número de casillas que ocupa
                \item Orientación: booleano que indica su posición.
            \end{itemize}
        \end{itemize}
    \end{block}
\end{frame}


\subsubsection{Mapas de disparo}

 \begin{frame}
     \frametitle{\insertsection}
     \framesubtitle{\hskip30pt \insertsubsection}
     \begin{block}{Mapas de disparos}
        \begin{itemize}
             \item Vectores de vectores (matrices)
             \item Representan los \textcolor{blue}{disparos que ha realizado el jugador} en una casilla del tablero del contrincante.
             \item Mismo tamaño que los tableros.
             \item Contienen la siguiente información:
             \begin{itemize}
                 \item 0: casilla sin descubrir.
                 \item 1: casilla de barco descubierta
                 \item -1: casilla de agua descubierta
                 \item -2: exclusión alrededor de barco hundido (solo es usado por los algoritmos de la máquina)
             \end{itemize}
        \end{itemize}
     \end{block}
 \end{frame}

 \subsubsection{Matriz de probabilidades}

 \begin{frame}
     \frametitle{\insertsection}
     \framesubtitle{\hskip30pt \insertsubsection}
     \begin{block}{Matriz de probabilidades}
         \begin{itemize}
             \item Vectores de vectores (matrices)
             \item Representa la \textcolor{blue}{probabilidad de cada casilla de contener un barco.}
             \item Mismo tamaño que los tableros.
             \item Solo es usado por el algoritmo Hunt/Target probabilístico
             \item Contiene por cada casilla la suma de los pesos.
         \end{itemize}
     \end{block}
 \end{frame}



\subsection{Estructura de archivos}

\begin{frame}
    \frametitle{\insertsection}
    \framesubtitle{\hskip30pt \insertsubsection}

    \begin{block}{}
        \renewcommand\DTstyle{\sffamily}
        \begin{small}
            
        
        \dirtree{%
        .1 \textbf{HundirFlota/}.
        .2 + archivo-tableros/ \\Archivador de tableros (generado).
        .2 + gui/ - Elementos de la interfaz.
        .3 botones.rkt.
        .3 canvas.rkt.
        .3 dialogos.rkt.
        .3 mensajes.rkt.
        .3 opciones.rkt.
        .3 paneles.rkt.
        .3 temporizadores.rkt.
        .3 textos.rkt.
        .3 ventana.rkt.
        }
    \end{small}
    \end{block}
\end{frame}

\begin{frame}
    \begin{block}{}
        \renewcommand\DTstyle{\sffamily}
        \begin{small}
            
        
        \dirtree{%
        .1 \textbf{HundirFlota/}.
        .2 + gui/.
        .3 + funcionesCallback - Funciones de los elementos gráficos.
        .4 callbackBotones.rkt.
        .4 callbackCanvas.rkt.
        .4 callbackDialogos.rkt.
        .4 callbackOpciones.rkt.
        .4 callbackTemporizadores.rkt.
        .4 callbackVentanas.rkt.
        }
    \end{small}
    \end{block}
\end{frame}


\begin{frame}
    \begin{block}{}
        \renewcommand\DTstyle{\sffamily}
        \begin{small}
            
        
        \dirtree{%
        .1 \textbf{HundirFlota/}.
        .2 funciones.rkt \\ Funciones variadas.
        .2 funcionesGUI.rkt.\\ Funciones variadas de GUI.
        .2 turno.rkt \\ Función cambio de turno.    
        }
    \end{small}
    \end{block}
\end{frame}

\begin{frame}
    \begin{block}{}
        \renewcommand\DTstyle{\sffamily}
        \begin{small}
            
        
        \dirtree{%
        .1 \textbf{HundirFlota/}.
        .2 funcionesLimpiar.rkt \\ Limpieza de canvases.
        .2 funcionesColocar.rkt \\ Colocación de barcos y su lógica.
        .2 funcionesDibujar.rkt \\ Dibujado en canvas.
        .2 funcionesDisparar.rkt \\ Lógica de disparar.
        .2 funcionesGenerar.rkt \\ Generación aleatoria en la colocación de barcos.
        .2 funcionesLogica.rkt \\ Lógica del juego.     
        }
    \end{small}
    \end{block}
\end{frame}

\begin{frame}
    \begin{block}{}
        \renewcommand\DTstyle{\sffamily}
        \begin{small}
            
        
        \dirtree{%
        .1 \textbf{HundirFlota/}.
        .2 + imagenes/ \\ Recursos gráficos.
        .2 + ayuda/ \\ Recursos de ayuda.
        .2 main.rkt \\ Programa principal.
        .2 estructuras.rkt \\ Estructuras principales.
        .2 macros.rkt \\ Macros globales.
        }
    \end{small}
    \end{block}
\end{frame}

